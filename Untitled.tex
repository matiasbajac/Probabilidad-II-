% Options for packages loaded elsewhere
\PassOptionsToPackage{unicode}{hyperref}
\PassOptionsToPackage{hyphens}{url}
%
\documentclass[
]{article}
\usepackage{amsmath,amssymb}
\usepackage{lmodern}
\usepackage{iftex}
\ifPDFTeX
  \usepackage[T1]{fontenc}
  \usepackage[utf8]{inputenc}
  \usepackage{textcomp} % provide euro and other symbols
\else % if luatex or xetex
  \usepackage{unicode-math}
  \defaultfontfeatures{Scale=MatchLowercase}
  \defaultfontfeatures[\rmfamily]{Ligatures=TeX,Scale=1}
\fi
% Use upquote if available, for straight quotes in verbatim environments
\IfFileExists{upquote.sty}{\usepackage{upquote}}{}
\IfFileExists{microtype.sty}{% use microtype if available
  \usepackage[]{microtype}
  \UseMicrotypeSet[protrusion]{basicmath} % disable protrusion for tt fonts
}{}
\makeatletter
\@ifundefined{KOMAClassName}{% if non-KOMA class
  \IfFileExists{parskip.sty}{%
    \usepackage{parskip}
  }{% else
    \setlength{\parindent}{0pt}
    \setlength{\parskip}{6pt plus 2pt minus 1pt}}
}{% if KOMA class
  \KOMAoptions{parskip=half}}
\makeatother
\usepackage{xcolor}
\IfFileExists{xurl.sty}{\usepackage{xurl}}{} % add URL line breaks if available
\IfFileExists{bookmark.sty}{\usepackage{bookmark}}{\usepackage{hyperref}}
\hypersetup{
  pdftitle={Parte II},
  pdfauthor={Matias Bajac},
  hidelinks,
  pdfcreator={LaTeX via pandoc}}
\urlstyle{same} % disable monospaced font for URLs
\usepackage[margin=1in]{geometry}
\usepackage{graphicx}
\makeatletter
\def\maxwidth{\ifdim\Gin@nat@width>\linewidth\linewidth\else\Gin@nat@width\fi}
\def\maxheight{\ifdim\Gin@nat@height>\textheight\textheight\else\Gin@nat@height\fi}
\makeatother
% Scale images if necessary, so that they will not overflow the page
% margins by default, and it is still possible to overwrite the defaults
% using explicit options in \includegraphics[width, height, ...]{}
\setkeys{Gin}{width=\maxwidth,height=\maxheight,keepaspectratio}
% Set default figure placement to htbp
\makeatletter
\def\fps@figure{htbp}
\makeatother
\setlength{\emergencystretch}{3em} % prevent overfull lines
\providecommand{\tightlist}{%
  \setlength{\itemsep}{0pt}\setlength{\parskip}{0pt}}
\setcounter{secnumdepth}{-\maxdimen} % remove section numbering
\ifLuaTeX
  \usepackage{selnolig}  % disable illegal ligatures
\fi

\title{Parte II}
\author{Matias Bajac}
\date{2023-04-14}

\begin{document}
\maketitle

\[\sigma\ algebra\]

\[ Def: \Omega\ conjunto\ , C\ familia\ de\ subconjuntos\ diremos\ que\ \Omega\ es\ una\ sigma\ algebra\ sii\]
\[propiedades\]

\[(i) A\ \in C\ \implies A^c\ \in C\]
\[(ii)\ \{A_n\}\ coleccion\ numerable\ "cerrado\ bajo\ uniones\ numerables" \implies \cup_{n=1}^{\infty}\ {A_n}\ \in\ C\]

\[ P(\Omega) = {\{subconjuntos\ de\ \Omega\}}\]
\[ si\ \Omega = \{a,b,c,d\}\]
\[C = \{\emptyset,{\{a,b}\},{\{c,d}\},{\Omega}\}\]

\[ ejemplo\ 3.2\]

\[ \Omega=\ \Omega^1\  \cup\ \Omega^2\ \cup\, \Omega^n\]

La familia de uniones de de \(\Omega^n\) es una \(\sigma\) algebra

\[C = \{\emptyset,{\Omega^1\ ,\Omega^2, \Omega^n}\}\]

\[Ejemplo\ 3.3\]

\[ Consideremos\ un\ conjunto\ \Omega\ arbitrario\,\ consideremos\ un\ \mathcal{A}\ la\ familia\ de\ conjuntos\ A\ \subset\ \Omega\ tales\ que\ el\ elemento\\ A\ es\]
\[numerable,\  o\  que\ su\ complemento\ es\ numerable.\ Entonces\ \mathcal{A}\ es\ \sigma\ algebra\]
\[ Si\ A\ \in\ \mathcal{A}\ \implies\ A^c\ \in\ \mathcal{A}\]

\[ si\ tenemos\  A_1\ numerable\ \implies\ A_1^c\ no\ numerable.\]
\[ Por Axioma\ 2\ \cup_{n=1}^{\infty} \ A_n^c\ \in\ \mathcal{A} = \cap_{n=1}^{\infty}\ A_n^c\ \subseteq\ A_1^c\]
\[ Entonces\ \mathcal{A}\ es\ \sigma\ algebra\] \[ ejemplo\ 3.4\]

\[ Observar\ que\ si\ \mathcal{A_\alpha}\ es\ una\ coleccion\ arbitraria\ de\ \sigma\ algebra\ en\ \Omega\, entonces\ \cap_{\alpha}\mathcal{A_\alpha}\ es\ tambien\ una\ \sigma\ algebra\]
\emph{Dem}: requiere probar cada uno de los tres axiomas.

Axioma 1: por hipótesis, \(\mathcal{A}_{\alpha}\) es \(\sigma\)-álgebra
\(\forall \alpha \in I \Rightarrow\) por axioma 1,
\(\Omega \in \mathcal{A}_{\alpha} \forall \alpha \in I \Rightarrow\) por
definición de intersección,
\(\Omega \in \bigcap\limits_{\alpha \in I} \mathcal{A}_{\alpha}\).

Axioma 2: si
\(A \in \bigcap\limits_{\alpha \in I} \mathcal{A}_{\alpha} \Rightarrow\)
por definición de intersección,
\(A \in \mathcal{A}_{\alpha} \forall \alpha \in I\). A su vez, si
\(A \in \mathcal{A}_{\alpha} \forall \alpha \in I \Rightarrow\) por
axioma 2,
\(A^c \in \mathcal{A}_{\alpha} \forall \alpha \in I \Rightarrow\) por
definición de intersección,
\(A^c \in \bigcap\limits_{\alpha \in I} \mathcal{A}_{\alpha}\).

Axioma 3: si
\(\{A_n\}_{n \in \mathbb{N}} \in \bigcap\limits_{\alpha \in I} \mathcal{A}_{\alpha} \Rightarrow\)
por definición de intersección,
\(\{A_n\}_{n \in \mathbb{N}} \subset \mathcal{A}_{\alpha} \forall \alpha \in I\).
Luego, si
\(\{A_n\}_{n \in \mathbb{N}} \subset \mathcal{A}_{\alpha} \forall \alpha \in I \Rightarrow\)
por definición de unión,
\(\{A_n\}_{n \in \mathbb{N}} \in \bigcup\limits_{\alpha \in I} \mathcal{A}_{\alpha}\).

\hypertarget{sigma-algebra-de-borel}{%
\section{sigma algebra de borel}\label{sigma-algebra-de-borel}}

\(B\mathcal(R)\) = \(\cap_{\alpha}\mathcal{A_{\alpha}}\),
\(\sigma - algebra\) siendo que~\(I\subset\mathcal{A_{\alpha}}\)

siendo que I = \{(a,b) : \(a,b \in \mathcal(R)\)\} , a \textless{} b\}

\[ ejemplo 3.7\]

La \(\sigma algebra\) de Borel tambien puede ser obtenida si cambiamos
la familia I por la famiuliad e intervalos semi abiertos (a,b{]} o
intervalos cerradas {[}a,b{]} o intervalos mas generales tipo
\((-\infty,b)\) \((\infty,b]\), variando a y b para todo R con a
\textless{} b. Basta probar que cada uno de estos nuevos intervalos
puede ser formado por elementos de \(B\mathcal(R)\)

Por ejemplo, en el ejercicio anterior, (0,1{]} =
\$\cup\_\{n\textgreater1\}~(0,1+1/n) \$ y por lo tanto
\((0,1] \in \mathcal{B}{\mathcal(R)}\). Tambien estan los conjuntos de
un elemento \{2\} \(\in \mathcal{B}{\mathcal(R)}\)

por ejemplo, en el ejercicio \((-\infty,1]\) = \(\cup_{n>1}\ (-n,1)\)

\hypertarget{teoria-de-la-medida}{%
\section{Teoria de la Medida}\label{teoria-de-la-medida}}

Un espacio \emph{medible} es un par \((\Omega,\mathcal{A})\) donde
\(\Omega\) es un conjunto, y \(\mathcal{A}\) es una \(\sigma\) algebra
de sobconjuntos de \(\Omega\)

Una medida es un espacio medible \((\Omega,\mathcal{A})\) es una funcion
\(\mu:\ \mathcal{A} \implies\ \mathcal{R}\) que se cumplas las
siguientes propiedades

\begin{enumerate}
\def\labelenumi{(\roman{enumi})}
\item
  \(\mu(A)\ \geq \mu(\emptyset)\), para todo \(A \in\ \mathcal{A}\)
\item
  \(A_{i}\) es una coleccion numerable y disjunta fe mienbros de
  \(mathcal{A}\), entonces \(\mu(\cup_{i}A_i)=\sum_{i}\mu(A_i)\)
\end{enumerate}

A la propiedad (ii) se la conoce como propiedad \(\sigma\) aditiva

si \(\mu(\Omega) = 1\), decimos que \(\mu\) es una medida de
probabilidad

Propiedades

\begin{enumerate}
\def\labelenumi{\arabic{enumi})}
\item
  \(A \in \mathcal{A} \implies\ P(A^c)=1-P(A)\)
\item
  Si A
\end{enumerate}

\end{document}
